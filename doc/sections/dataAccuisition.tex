\section{Data Acquisition}
\label{sec:dataset}

The first step is to acquire the necessary data to train the model. For this project, voice data from the Mozilla Common Voice project is used. Since the dataset is quite large, one of the smaller subsets was chosen for prototyping and feature selection. Specifically, the Common Voice Delta Segment 16.1 dataset, which is approximately 1.7 GB, was selected. 

\subsection{Structure}
The Mozilla Common Voice dataset includes a TSV file containing labels and the paths to audio files. The dataset structure is as follows:


{\iosevka
\begin{verbatim}
data
|-- cv-corpus-16.1-delta-2023-12-06
|-- en
    |-- clip_durations.tsv
    |-- clips (folder)
        |-- ...
        |-- ...
    |-- invalidated.tsv
    |-- other.tsv
    |-- reported.tsv
    |-- validated.tsv
\end{verbatim}
}



\subsection{Filtering}
Not all data in the dataset is required for this project. Filtering is applied to retain only the necessary features: gender, age, the spoken sentence, and the audio file path. Additionally, some data entries were found to have missing or incorrect labels. Therefore, the data is further filtered to exclude entries with incomplete or incorrect labels. The filtered features are shown in Table~\ref{tab:filtered_features}.


\begin{table}[h]
    \centering % Centers the table
    \begin{adjustbox}{max width=\textwidth}
        \begin{tabular}{llll}
\toprule
path & sentence & age & gender \\
\midrule
common\_voice\_en\_38570599.mp3 & Flatow was given a co-star credit. & twenties & male \\
common\_voice\_en\_39572424.mp3 & A timetable for the move was not specified. & fifties & male \\
common\_voice\_en\_39548774.mp3 & Where can i find the supreme leader? & teens & female \\
common\_voice\_en\_38594742.mp3 & Down jumped the driver, and out got Mr. Pickwick. & twenties & male \\
common\_voice\_en\_38514380.mp3 & He was the original sponsor of legislation creating pension reform in New Jersey. & sixties & male \\
\bottomrule
\end{tabular}
  % Imports the LaTeX table from an external file
    \end{adjustbox}
    \caption{Table of filtered features}
    \label{tab:filtered_features}
\end{table}


\subsection{Balancing}
Once the features are filtered, the dataset is checked for balance across labels. In this project, balance means an equal distribution of male and female samples. An imbalanced dataset could cause the model to be biased toward the majority class. If necessary, the excess data from the majority class is removed to achieve balance.

\paragraph{}
In this case, the gender distribution is 1528 male and 1498 female samples, which is nearly balanced (close to a 50:50 ratio). This distribution is sufficient for the deep learning model to learn effectively. However, there is a third category, "other," which has only 50 samples. This label does not provide useful information for this task and is excluded. These samples could, however, be utilized later for unseen data testing with human validation.

\begin{figure}[h] % 'h' option places the figure approximately here
    \centering % Center the image
    \includegraphics[width=0.8\textwidth]{images/gender_features.png}
    \caption{Distribution of the gender data} % Optional caption
    \label{fig:gender_distribution} % Optional label for referencing
\end{figure}