\section{Data accuisition}

    The first step is to accuire the necessery data to train the model. For this project the voice data from the Mozilla Common Voice project is used. Since most of the data is fairly large it was decided to use one of the smaller datasets, atleast for prototyping and feature selection. For that reason the dataset Common Voice Delta Segment 16.1 was chosen. The dataset consists of roughly 1.7 GB of data. First let's take a look at the structure of the dataset. For this the Mozilla Common Voice dataset contains a tsv file containing the labels and the path to the audio files. 
    \break
    The dataset is structured as follows:
    
    {\iosevka
    \begin{verbatim}
    data
    |-- cv-corpus-16.1-delta-2023-12-06
        |-- en
            |-- clip_durations.tsv
            |-- clips (folder)
                |-- ...
                |-- ...
            |-- invalidated.tsv
            |-- other.tsv
            |-- reported.tsv
            |-- validated.tsv
    \end{verbatim}
    }

    \paragraph{Data Filtering}
    since not all of the data is needed for this project, the data is filtered to only contain the necessary data. For the case of the project, only the gender, age, the spoken sentence and the path to the audio file is kept. After a brief look at the data, it showed that some of the data is not labeled correctly. Some gender labels are missing. For this reason the data is filtered to only contain the data with the correct labels.

    \begin{table}[h!]
        \begin{adjustbox}{max width=\textwidth}
        \begin{tabular}{llr}
\toprule
path & age & gender \\
\midrule
common\_voice\_en\_38510811.mp3 & sixties & 0 \\
common\_voice\_en\_39352286.mp3 & fourties & 0 \\
common\_voice\_en\_38529039.mp3 & thirties & 0 \\
common\_voice\_en\_38627386.mp3 & thirties & 0 \\
common\_voice\_en\_38498741.mp3 & thirties & 0 \\
\bottomrule
\end{tabular}
  % Imports the LaTeX table from the file
        \end{adjustbox}
        \caption{Table of filtered features}
    \end{table}

    \paragraph{Data Balancing}
    After the necessary features are selected it should be checked if the data is balanced. This means the data should contain the same amount of data for each label. In the case of this project, this means the data should contain the same amount of clipes spoken by male and female subjects. If the data is not balanced, the model could be biased towards the majority class. In this case the data is balanced by removing the excess data from the majority class.

    \begin{figure}[h] % 'h' option places the figure approximately here
        \centering % Center the image
	    \includegraphics[width=0.5\textwidth]{images/gender_features.png}
        \caption{distribution of the gender data} % Optional caption
        \label{fig:gender_distribution} % Optional label for referencing
    \end{figure}


    In this case the gender distribution is with XXX male and XXX female almost 0.5. This should be good enough for the deep learning model to learn without any complication. In case of insufficient results this topic will get reevaluated. Another issue is, that there is a third categorie "others". This label does obviously not provide any useful information which would allow for any performance benefits. For that reason this data has to get discared aswell.
