\section{Challanges}
\label{sec:challanges}
No project comes without any complications and challenges. The main confronting challenges were the lack of Intuitiveness and the experimental approach it takes to train a neural network.

\subsection{Intuitiveness}
Training neural networks can be challenging because the learning process is not always intuitive. Neural networks are complex systems with many layers and parameters, making it hard to understand how they learn from data. Problems like poor performance or overfitting are often difficult to diagnose because the inner workings of the model are not directly visible. Tasks like adjusting hyperparameters, such as the learning rate or regularization, require trial and error, as their effects on training are not always predictable. This lack of transparency means practitioners must rely on experience, visualizations, and experimentation to understand and improve model behavior.

\subsection{Experimental}
Training neural networks is highly experimental, as there is no universal formula for success. Each dataset and problem often requires different architectures, preprocessing methods, and hyperparameter settings. Even small changes in data or configuration can lead to unpredictable results, requiring extensive testing to find the best approach. This trial-and-error process can be time-consuming and computationally expensive, especially when experimenting with complex models or large datasets. While tools and frameworks can assist, the need for experimentation remains a significant challenge in neural network training.